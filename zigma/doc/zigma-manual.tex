%% Zigma LaTeX Class - User Manual
%% Comprehensive documentation for v0.9.0
\documentclass[11pt,twoside,openright]{book}

% Packages
\usepackage[utf8]{inputenc}
\usepackage[T1]{fontenc}
\usepackage{lmodern}
\usepackage[margin=1in]{geometry}
\usepackage{xcolor}
\usepackage{listings}
\usepackage{hyperref}
\usepackage{graphicx}
\usepackage{fancyhdr}
\usepackage{titlesec}
\usepackage{tcolorbox}
\usepackage{enumitem}
\usepackage{multicol}

% Colors
\definecolor{zigmablue}{RGB}{0,128,128}
\definecolor{codebg}{RGB}{245,245,245}
\definecolor{codeframe}{RGB}{200,200,200}

% Hyperref setup
\hypersetup{
  colorlinks=true,
  linkcolor=zigmablue,
  urlcolor=zigmablue,
  citecolor=zigmablue,
  bookmarksnumbered=true,
  pdftitle={Zigma LaTeX Class - User Manual v0.9.0},
  pdfauthor={Emmanuele Somma},
  pdfsubject={LaTeX Document Class},
  pdfkeywords={LaTeX, expl3, document class, academic publishing}
}

% Listings setup for LaTeX code
\lstdefinestyle{latexcode}{
  language=[LaTeX]TeX,
  basicstyle=\ttfamily\small,
  backgroundcolor=\color{codebg},
  frame=single,
  framerule=0.5pt,
  rulecolor=\color{codeframe},
  numbers=left,
  numberstyle=\tiny\color{gray},
  breaklines=true,
  breakatwhitespace=true,
  tabsize=2,
  showstringspaces=false,
  keywordstyle=\color{blue}\bfseries,
  commentstyle=\color{gray}\itshape,
  stringstyle=\color{red},
  xleftmargin=2em,
  xrightmargin=1em,
  aboveskip=1em,
  belowskip=1em
}

\lstset{style=latexcode}

% Custom commands
\newcommand{\zigma}{\textbf{Zigma}}
\newcommand{\cmd}[1]{\texttt{\textbackslash#1}}
\newcommand{\pkg}[1]{\textsf{#1}}
\newcommand{\opt}[1]{\texttt{#1}}
\newcommand{\meta}[1]{$\langle$\textit{#1}$\rangle$}

% Headers and footers
\pagestyle{fancy}
\fancyhf{}
\fancyhead[LE,RO]{\thepage}
\fancyhead[LO]{\nouppercase{\rightmark}}
\fancyhead[RE]{\nouppercase{\leftmark}}
\renewcommand{\headrulewidth}{0.4pt}

% Title formatting
\titleformat{\chapter}[display]
  {\normalfont\huge\bfseries\color{zigmablue}}
  {\chaptertitlename\ \thechapter}{20pt}{\Huge}
\titleformat{\section}
  {\normalfont\Large\bfseries\color{zigmablue}}
  {\thesection}{1em}{}
\titleformat{\subsection}
  {\normalfont\large\bfseries}
  {\thesubsection}{1em}{}

% Colored boxes for examples and notes
\tcbset{
  examplebox/.style={
    colback=zigmablue!5,
    colframe=zigmablue,
    title=Example,
    fonttitle=\bfseries,
    boxrule=0.5pt
  },
  notebox/.style={
    colback=yellow!10,
    colframe=orange,
    title=Note,
    fonttitle=\bfseries,
    boxrule=0.5pt
  },
  warningbox/.style={
    colback=red!5,
    colframe=red,
    title=Warning,
    fonttitle=\bfseries,
    boxrule=0.5pt
  }
}

% Document metadata
\title{%
  {\Huge\textbf{\zigma{} LaTeX Class}}\\[0.5cm]
  {\Large User Manual and Reference Guide}\\[0.3cm]
  {\large Version 0.9.0}
}
\author{Emmanuele Somma\\[0.2cm]\texttt{emmanuele@exedre.org}}
\date{November 7, 2025}

\begin{document}

%% ====================================================================
%% TITLE PAGE
%% ====================================================================

\frontmatter

\maketitle

\clearpage

%% ====================================================================
%% COPYRIGHT AND LICENSE
%% ====================================================================

\thispagestyle{empty}
\vspace*{\fill}
\begin{center}
\textbf{Zigma LaTeX Class v0.9.0}

Copyright \textcopyright\ 2025 Emmanuele Somma

\bigskip

This work may be distributed and/or modified under the
conditions of the LaTeX Project Public License, either version 1.3c
of this license or (at your option) any later version.
The latest version of this license is in:

\url{http://www.latex-project.org/lppl.txt}

and version 1.3c or later is part of all distributions of LaTeX
version 2008 or later.

\bigskip

\textbf{Repository:} \url{https://git.xed.it/exedre/zigma-class}

\bigskip

This manual documents version 0.9.0 of the Zigma LaTeX class.
\end{center}
\vspace*{\fill}

\clearpage

%% ====================================================================
%% TABLE OF CONTENTS
%% ====================================================================

\tableofcontents

\clearpage

%% ====================================================================
%% PREFACE
%% ====================================================================

\chapter*{Preface}
\addcontentsline{toc}{chapter}{Preface}

\zigma{} is a modern, flexible LaTeX document class designed for academic journals and scientific publications. Built entirely with LaTeX3's \pkg{expl3} programming layer, it provides a robust foundation for professional document preparation with extensive customization capabilities.

\section*{What Makes Zigma Special?}

\begin{itemize}[leftmargin=*]
  \item \textbf{Modern Implementation}: Pure \pkg{expl3} codebase ensures reliability and maintainability
  \item \textbf{Modular Architecture}: Plugin system supports multiple base classes (article, memoir, KOMA-Script, rho)
  \item \textbf{Smart Features}: Intelligent cross-referencing, automatic ORCID integration, flexible bibliography support
  \item \textbf{User-Friendly}: Simple key-value configuration interface via \cmd{zigmasetup}
  \item \textbf{Production-Ready}: Extensively tested with comprehensive documentation
\end{itemize}

\section*{Who Should Use Zigma?}

This class is ideal for:

\begin{itemize}[leftmargin=*]
  \item Academic researchers writing journal articles
  \item PhD students preparing dissertations
  \item Conference organizers needing consistent formatting
  \item Journal editors seeking a flexible submission system
  \item Anyone requiring professional academic document preparation
\end{itemize}

\section*{Manual Structure}

This manual is organized into the following parts:

\begin{description}
  \item[Part I: Getting Started] Installation, quick start guide, and basic usage
  \item[Part II: Core Features] Author management, metadata, bibliography
  \item[Part III: Advanced Features] Cross-referencing, templates, customization
  \item[Part IV: Reference] Complete key reference, command reference
  \item[Part V: Appendices] Examples, troubleshooting, development
\end{description}

\section*{Conventions Used in This Manual}

\begin{itemize}[leftmargin=*]
  \item \cmd{command} represents a LaTeX command
  \item \opt{option} represents a class option or key
  \item \meta{argument} represents a placeholder for user input
  \item Code examples appear in \colorbox{codebg}{\texttt{monospaced gray boxes}}
\end{itemize}

\vfill

\begin{center}
\textit{Made with ❤️ and expl3 by humans and AI working together}
\end{center}

\mainmatter

%% ====================================================================
%% PART I: GETTING STARTED
%% ====================================================================

\part{Getting Started}

\chapter{Introduction}

\section{Overview}

The \zigma{} LaTeX class provides a comprehensive framework for creating professional academic documents. Version 0.9.0 introduces advanced features including smart cross-referencing with automatic type detection and extensive bibliography support with 15 citation styles.

\subsection{Key Features}

\begin{description}
  \item[Multi-Author Support] Robust system for managing multiple authors with affiliations, ORCID integration, and corresponding author markers
  
  \item[Base Class System] Modular plugins for different document types:
  \begin{itemize}
    \item \textbf{Article}: Standard LaTeX articles
    \item \textbf{Memoir}: Books, theses, long documents
    \item \textbf{KOMA-Script}: European typography (scrartcl, scrreprt, scrbook)
    \item \textbf{Rho}: Academic journals with full metadata support
  \end{itemize}
  
  \item[Smart Cross-References] Intelligent referencing with automatic type detection for 11 label types (sections, figures, tables, equations, etc.)
  
  \item[Bibliography Integration] 15 citation styles including IEEE, APA, Chicago, Nature, Harvard, Vancouver, MLA, and more
  
  \item[Template System] Pre-configured templates for common publication types (IEEE, APA, Nature, thesis)
  
  \item[Customization] Extensive configuration via key-value interface with support for headers, footers, colors, fonts, and layouts
\end{description}

\subsection{System Requirements}

\begin{itemize}
  \item \textbf{LaTeX Engine}: LuaLaTeX (recommended), XeLaTeX, or pdfLaTeX
  \item \textbf{TeX Distribution}: TeX Live 2022 or later, MiKTeX 2022 or later
  \item \textbf{Required Packages}: expl3, xparse, hyperref, xcolor
  \item \textbf{Optional Packages}: biblatex (for bibliography), academicons (for ORCID icons)
\end{itemize}

\section{Installation}

\subsection{Quick Installation}

The simplest way to use \zigma{} is to clone the repository and reference it directly:

\begin{lstlisting}
git clone https://git.xed.it/exedre/zigma-class.git
cd zigma-class
\end{lstlisting}

Then in your document:

\begin{lstlisting}
\documentclass{zigma-class/zigma}
\end{lstlisting}

\subsection{Manual Installation}

For system-wide installation:

\begin{enumerate}
  \item Download the latest release from the repository
  \item Copy \texttt{zigma-class/} directory to your local texmf tree:
  \begin{itemize}
    \item Linux/Mac: \texttt{\textasciitilde/texmf/tex/latex/}
    \item Windows: \texttt{C:\textbackslash Users\textbackslash\meta{username}\textbackslash texmf\textbackslash tex\textbackslash latex\textbackslash}
  \end{itemize}
  \item Run \texttt{texhash} or \texttt{mktexlsr} to update the filename database
\end{enumerate}

\subsection{Verifying Installation}

Create a minimal test file:

\begin{lstlisting}
\documentclass{zigma-class/zigma}
\zigmasetup{title = {Test Document}}
\begin{document}
\maketitle
Hello, Zigma!
\end{document}
\end{lstlisting}

Compile with:

\begin{lstlisting}[language=bash]
lualatex test.tex
\end{lstlisting}

If compilation succeeds, \zigma{} is correctly installed.

\chapter{Quick Start Guide}

\section{Your First Document}

Let's create a simple academic article:

\begin{lstlisting}
\documentclass{zigma-class/zigma}

\zigmasetup{
  title = {Introduction to Quantum Computing},
  subtitle = {A Beginner's Guide},
  
  affiliations.0 = {MIT - Department of Computer Science},
  
  authors.0.name = {Alice Johnson},
  authors.0.email = {alice@mit.edu},
  authors.0.orcid = {0000-0001-2345-6789},
  authors.0.affiliation = {0},
  authors.0.corresponding = {true},
  
  abstract = {This paper provides an accessible introduction 
              to quantum computing for beginners...},
  keywords = {quantum computing, qubits, algorithms},
}

\begin{document}

\maketitle

\section{Introduction}
\label{sec:intro}

Quantum computing represents a paradigm shift...

\section{Quantum Bits}
\label{sec:qubits}

As discussed in \zigmaref{sec:intro}, quantum bits differ...

\end{document}
\end{lstlisting}

\section{Compilation}

Compile your document with LuaLaTeX:

\begin{lstlisting}[language=bash]
lualatex mydocument.tex
\end{lstlisting}

For documents with cross-references, compile twice:

\begin{lstlisting}[language=bash]
lualatex mydocument.tex
lualatex mydocument.tex
\end{lstlisting}

For documents with bibliography, use:

\begin{lstlisting}[language=bash]
lualatex mydocument.tex
biber mydocument
lualatex mydocument.tex
lualatex mydocument.tex
\end{lstlisting}

Or simply use \texttt{latexmk}:

\begin{lstlisting}[language=bash]
latexmk -lualatex -interaction=nonstopmode mydocument.tex
\end{lstlisting}

\section{Understanding the Output}

Your compiled PDF will include:

\begin{itemize}
  \item Title (full-width in two-column mode)
  \item Subtitle (if specified)
  \item Author name with ORCID icon (clickable)
  \item Author email (clickable mailto: link)
  \item Affiliation with superscript marker
  \item Abstract (full-width, bold label)
  \item Keywords (formatted inline list)
  \item Corresponding author footer (automatic)
  \item Your document content
\end{itemize}

%% ====================================================================
%% PART II: CORE FEATURES
%% ====================================================================

\part{Core Features}

\chapter{Document Configuration}

\section{The \texttt{\textbackslash zigmasetup} Command}

All document configuration in \zigma{} is done via the \cmd{zigmasetup} command with key-value pairs:

\begin{lstlisting}
\zigmasetup{
  key1 = {value1},
  key2 = {value2},
  nested.key = {value},
}
\end{lstlisting}

\subsection{Configuration Timing}

\begin{tcolorbox}[notebox]
\cmd{zigmasetup} can be called multiple times and in different locations:
\begin{itemize}
  \item In the preamble (before \cmd{begin\{document\}})
  \item After \cmd{begin\{document\}}
  \item Multiple times to override previous settings
\end{itemize}
\end{tcolorbox}

\section{Basic Metadata}

\subsection{Title and Subtitle}

\begin{lstlisting}
\zigmasetup{
  title = {The Main Title of Your Document},
  subtitle = {An Optional Subtitle},
}
\end{lstlisting}

\subsection{Date}

\begin{lstlisting}
\zigmasetup{
  date = {\today},         % Default: today's date
  date.show = {true},      % Show/hide date (default: true)
}
\end{lstlisting}

\subsection{Abstract and Keywords}

\begin{lstlisting}
\zigmasetup{
  abstract = {This paper investigates...},
  keywords = {keyword1, keyword2, keyword3},
}
\end{lstlisting}

\section{Clickable Titles and URLs}

Make your title a clickable hyperlink:

\begin{lstlisting}
\zigmasetup{
  title = {My Research Paper},
  title.url = {https://doi.org/10.1234/paper},
  title.url.show = {true},  % Optional: show URL in footer
}
\end{lstlisting}

\chapter{Author Management}

\section{Single Author}

For a document with one author:

\begin{lstlisting}
\zigmasetup{
  affiliations.0 = {University of Example},
  
  authors.0.name = {John Doe},
  authors.0.email = {john.doe@example.edu},
  authors.0.orcid = {0000-0001-2345-6789},
  authors.0.affiliation = {0},
  authors.0.corresponding = {true},
}
\end{lstlisting}

\section{Multiple Authors}

For documents with multiple authors:

\begin{lstlisting}
\zigmasetup{
  % Define affiliations first
  affiliations.0 = {MIT},
  affiliations.1 = {Stanford University},
  affiliations.2 = {CERN},
  
  % First author
  authors.0.name = {Alice Smith},
  authors.0.email = {alice@mit.edu},
  authors.0.orcid = {0000-0001-1111-1111},
  authors.0.affiliation = {0,2},  % Multiple affiliations
  authors.0.corresponding = {true},
  
  % Second author
  authors.1.name = {Bob Jones},
  authors.1.affiliation = {1},
  
  % Third author
  authors.2.name = {Carol White},
  authors.2.email = {carol@cern.ch},
  authors.2.affiliation = {2},
}
\end{lstlisting}

\section{Author Properties}

\begin{description}
  \item[\opt{authors.N.name}] (Required) Author's full name
  \item[\opt{authors.N.email}] Email address (creates clickable link)
  \item[\opt{authors.N.orcid}] ORCID identifier (displays icon with link)
  \item[\opt{authors.N.affiliation}] Comma-separated affiliation IDs
  \item[\opt{authors.N.corresponding}] \opt{true} or \opt{false} (default: false)
\end{description}

\subsection{ORCID Integration}

When an ORCID is provided, \zigma{} automatically:

\begin{itemize}
  \item Displays the ORCID icon (from \pkg{academicons} package)
  \item Makes it clickable to \texttt{https://orcid.org/}\meta{id}
  \item Positions it as superscript after the author name
\end{itemize}

\subsection{Corresponding Author Markers}

Corresponding authors are indicated with configurable symbols:

\begin{lstlisting}
\zigmasetup{
  corresponding.marker = {envelope},  % envelope, star, or asterisk
  corresponding.show-footer = {true}, % Show email footer
  
  authors.0.corresponding = {true},
}
\end{lstlisting}

Available marker symbols:
\begin{itemize}
  \item \opt{envelope}: ✉ (default, from \pkg{marvosym} package)
  \item \opt{star}: ★ (mathematical star)
  \item \opt{asterisk}: * (simple asterisk)
\end{itemize}

The footer automatically lists all corresponding authors' emails:

\begin{quote}
✉ Corresponding author: alice@mit.edu
\end{quote}

Or for multiple corresponding authors:

\begin{quote}
✉ Corresponding authors: alice@mit.edu, bob@stanford.edu
\end{quote}

\chapter{Metadata and DOI}

\section{Publication Metadata}

Add metadata footer with DOI, dates, and license:

\begin{lstlisting}
\zigmasetup{
  metadata.doi = {10.1234/journal.2024.001},
  metadata.url = {https://example.org/paper},
  metadata.url.show = {true},
  
  metadata.received = {2024-01-15},
  metadata.revised = {2024-03-10},
  metadata.accepted = {2024-05-20},
  metadata.published = {2024-06-01},
  
  metadata.license = {CC BY 4.0},
}
\end{lstlisting}

This generates a professional metadata footer:

\begin{quote}
\rule{\textwidth}{0.4pt}

\textbf{DOI:} \textcolor{blue}{10.1234/journal.2024.001} $\vert$ 
Received: 2024-01-15 $\vert$ 
Accepted: 2024-05-20

Published: 2024-06-01

\textbf{License:} CC BY 4.0

\rule{\textwidth}{0.4pt}
\end{quote}

\section{Journal Information}

For journal publications:

\begin{lstlisting}
\zigmasetup{
  journal.name = {Nature Physics},
  journal.url = {https://nature.com/nphys},
  journal.url.show = {true},
}
\end{lstlisting}

\chapter{Bibliography}

\section{Basic Bibliography Setup}

Enable bibliography support with class options:

\begin{lstlisting}
\documentclass[bib,bibfile=references.bib]{zigma-class/zigma}

\zigmasetup{
  bib.style = {ieee},  % or apa, chicago, nature, etc.
}
\end{lstlisting}

\section{Citation Styles}

\zigma{} supports 15 citation styles:

\begin{multicols}{2}
\begin{enumerate}
  \item ieee
  \item apa
  \item chicago
  \item nature
  \item numeric
  \item authoryear
  \item harvard
  \item vancouver
  \item mla
  \item alphabetic
  \item verbose
  \item trad-abbrv
  \item acm
  \item authoryear-comp
  \item philosophy
\end{enumerate}
\end{multicols}

\subsection{Quick Style Selection}

\begin{lstlisting}
% Option 1: Class option
\documentclass[bibpreset=harvard]{zigma-class/zigma}

% Option 2: zigmasetup
\zigmasetup{bib.style = {harvard}}
\end{lstlisting}

\section{Citation Commands}

\begin{description}
  \item[\cmd{cite\{\meta{key}\}}] Standard citation
  \item[\cmd{textcite\{\meta{key}\}}] In-text citation (author-year styles)
  \item[\cmd{parencite\{\meta{key}\}}] Parenthetical citation
  \item[\cmd{footcite\{\meta{key}\}}] Footnote citation (verbose styles)
\end{description}

\section{Multiple Bibliography Files}

Support for multiple .bib files:

\begin{lstlisting}
\documentclass[bibfile={refs1.bib,refs2.bib,refs3.bib}]{zigma-class/zigma}
\end{lstlisting}

\section{Advanced Features}

\subsection{Per-Chapter Bibliographies}

For books and theses:

\begin{lstlisting}
\begin{refsection}
  \chapter{Introduction}
  Content with \cite{ref1}...
  \printbibliography[heading=subbibliography]
\end{refsection}

\begin{refsection}
  \chapter{Methods}
  Content with \cite{ref2}...
  \printbibliography[heading=subbibliography]
\end{refsection}
\end{lstlisting}

\subsection{Split Bibliographies}

Separate primary and secondary sources:

\begin{lstlisting}
% In .bib file, add keywords:
@article{source1,
  author = {...},
  keywords = {primary},
}

% In document:
\printbibliography[keyword=primary,title={Primary Sources}]
\printbibliography[keyword=secondary,title={Secondary Literature}]
\end{lstlisting}

%% ====================================================================
%% PART III: ADVANCED FEATURES
%% ====================================================================

\part{Advanced Features}

\chapter{Smart Cross-References}

\section{Introduction}

Version 0.9.0 introduces intelligent cross-referencing with automatic type detection. No need to remember if you're referencing a section, figure, or table---\zigma{} detects it from the label prefix.

\section{Basic Usage}

\subsection{Standard References}

\begin{lstlisting}
\section{Introduction}
\label{sec:intro}

\begin{figure}
  \caption{Results}
  \label{fig:results}
\end{figure}

\begin{table}
  \caption{Data}
  \label{tab:data}
\end{table}

% Smart references (auto-detect type)
See \zigmaref{sec:intro} for background.
The \zigmaref{fig:results} shows the findings.
Values in \zigmaref{tab:data} confirm this.
\end{lstlisting}

Output:
\begin{quote}
See Section 1 for background.\\
The Figure 1 shows the findings.\\
Values in Table 1 confirm this.
\end{quote}

\subsection{Multiple References}

\begin{lstlisting}
% Oxford comma formatting
See \zigmarefs{sec:intro,sec:methods,sec:results}.
Figures \zigmarefs{fig:a,fig:b,fig:c} illustrate this.
\end{lstlisting}

Output:
\begin{quote}
See Sections 1, 2, and 3.\\
Figures 1, 2, and 3 illustrate this.
\end{quote}

\section{Reference Commands}

\begin{description}
  \item[\cmd{zigmaref\{\meta{label}\}}] Smart reference with prefix\\
    Example: \texttt{\cmd{zigmaref\{sec:intro\}}} → "Section 1"
  
  \item[\cmd{Zigmaref\{\meta{label}\}}] Uppercase variant\\
    Example: \texttt{\cmd{Zigmaref\{fig:plot\}}} → "FIGURE 2"
  
  \item[\cmd{zigmaref*\{\meta{label}\}}] Number only (no prefix)\\
    Example: \texttt{\cmd{zigmaref*\{sec:intro\}}} → "1"
  
  \item[\cmd{zigmarefs\{\meta{lab1,lab2,...}\}}] Multiple references\\
    Example: \texttt{\cmd{zigmarefs\{sec:a,sec:b\}}} → "Sections 1 and 2"
  
  \item[\cmd{zigmapageref\{\meta{label}\}}] Page reference\\
    Example: \texttt{\cmd{zigmapageref\{sec:intro\}}} → "on page 5"
  
  \item[\cmd{zigmafullref\{\meta{label}\}}] Full reference\\
    Example: \texttt{\cmd{zigmafullref\{fig:plot\}}} → "Figure 2 on page 12"
\end{description}

\section{Supported Label Types}

\zigma{} auto-detects 11 label types:

\begin{center}
\begin{tabular}{ll}
\textbf{Prefix} & \textbf{Type} \\
\hline
\texttt{sec:} & Section \\
\texttt{ch:} & Chapter \\
\texttt{fig:} & Figure \\
\texttt{tab:} & Table \\
\texttt{eq:} & Equation \\
\texttt{lst:} & Listing \\
\texttt{alg:} & Algorithm \\
\texttt{thm:} & Theorem \\
\texttt{lem:} & Lemma \\
\texttt{def:} & Definition \\
\texttt{app:} & Appendix \\
\end{tabular}
\end{center}

\section{Multilingual Support}

Customize labels for any language:

\begin{lstlisting}
% Italian
\zigmasetup{
  labels.section = {Sezione},
  labels.sections = {Sezioni},
  labels.figure = {Figura},
  labels.figures = {Figure},
  labels.table = {Tabella},
  labels.tables = {Tabelle},
}
\end{lstlisting}

Now \texttt{\cmd{zigmaref\{sec:intro\}}} outputs "Sezione 1" instead of "Section 1".

\subsection{Available Label Keys}

All labels come in singular and plural forms:

\begin{multicols}{2}
\begin{itemize}[noitemsep]
  \item \opt{labels.section / sections}
  \item \opt{labels.chapter / chapters}
  \item \opt{labels.figure / figures}
  \item \opt{labels.table / tables}
  \item \opt{labels.equation / equations}
  \item \opt{labels.listing / listings}
  \item \opt{labels.algorithm / algorithms}
  \item \opt{labels.theorem / theorems}
  \item \opt{labels.lemma / lemmas}
  \item \opt{labels.definition / definitions}
  \item \opt{labels.appendix / appendices}
  \item \opt{labels.page / pages}
\end{itemize}
\end{multicols}

\chapter{Base Class System}

\section{Overview}

\zigma{}'s base class system allows you to choose the underlying LaTeX class while maintaining consistent features. Four base classes are available:

\begin{enumerate}
  \item \textbf{Article}: Standard LaTeX article class (default for most)
  \item \textbf{Memoir}: For books, theses, and long documents
  \item \textbf{KOMA-Script}: European typography standards (scrartcl, scrreprt, scrbook)
  \item \textbf{Rho}: Academic journal style with full metadata support (default)
\end{enumerate}

\section{Selecting a Base Class}

\begin{lstlisting}
% Use article base class
\documentclass[base=article]{zigma-class/zigma}

% Use memoir for a thesis
\documentclass[base=memoir]{zigma-class/zigma}

% Use KOMA-Script for European style
\documentclass[base=scrartcl]{zigma-class/zigma}

% Use rho for journal articles (default)
\documentclass{zigma-class/zigma}  % or base=rho
\end{lstlisting}

\section{Base Class Comparison}

\begin{center}
\begin{tabular}{p{2.5cm}p{2.5cm}p{2.5cm}p{2.5cm}}
\textbf{Feature} & \textbf{Article} & \textbf{Memoir} & \textbf{KOMA} \\
\hline
Two-column & Yes & No & Yes \\
Chapters & No & Yes & Yes* \\
Typography & American & Book & European \\
Font style & Serif & Serif & Sans (titles) \\
Use case & Papers & Theses & Papers \\
Margins & Standard & Book & Configurable \\
\multicolumn{4}{l}{*scrreprt and scrbook only}
\end{tabular}
\end{center}

\section{Base Class Features}

\subsection{Article Base Class}

\begin{itemize}
  \item Standard LaTeX article class
  \item Two-column support
  \item Compact layout
  \item Full-width title, abstract, keywords
  \item Best for: Journal papers, conference proceedings
\end{itemize}

\subsection{Memoir Base Class}

\begin{itemize}
  \item Memoir class for long documents
  \item Single column default
  \item Chapter support
  \item Larger fonts and spacing
  \item Book-style typography
  \item Best for: PhD theses, technical books, monographs
\end{itemize}

\subsection{KOMA-Script Base Classes}

\begin{itemize}
  \item Three classes: scrartcl, scrreprt, scrbook
  \item European typography standards
  \item Sans-serif titles
  \item Small fonts for abstract/keywords
  \item Highly configurable via \cmd{KOMAoptions}
  \item Best for: European academic publications
\end{itemize}

\subsection{Rho Base Class}

\begin{itemize}
  \item Academic journal style
  \item Full metadata support (DOI, dates, license)
  \item Customizable headers and footers
  \item Two-column layout
  \item Professional journal appearance
  \item Best for: Journal submissions, online publications
\end{itemize}

\chapter{Templates}

\section{Template System}

Templates provide pre-configured settings for common publication types. They set base class, colors, fonts, bibliography style, and layout automatically.

\section{Available Templates}

\subsection{IEEE Template}

\begin{lstlisting}
\documentclass[template=ieee,bibfile=refs.bib]{zigma-class/zigma}
\end{lstlisting}

Features:
\begin{itemize}
  \item Two-column article format
  \item IEEE bibliography style
  \item 0.75-inch margins
  \item Standard IEEE conference paper appearance
\end{itemize}

\subsection{APA Template}

\begin{lstlisting}
\documentclass[template=apa,bibfile=refs.bib]{zigma-class/zigma}
\end{lstlisting}

Features:
\begin{itemize}
  \item APA 6th edition style
  \item Author-year citations
  \item Standard margins
  \item Psychology/social sciences formatting
\end{itemize}

\subsection{Nature Template}

\begin{lstlisting}
\documentclass[template=nature,bibfile=refs.bib]{zigma-class/zigma}
\end{lstlisting}

Features:
\begin{itemize}
  \item Nature journal style
  \item Numeric citations
  \item Compact formatting
  \item Science publication standards
\end{itemize}

\subsection{Thesis Template}

\begin{lstlisting}
\documentclass[template=thesis]{zigma-class/zigma}
\end{lstlisting}

Features:
\begin{itemize}
  \item Memoir base class
  \item Chapter support
  \item Larger fonts (12pt)
  \item Book-style layout
  \item Per-chapter bibliographies
\end{itemize}

\subsection{Il Cibernetico Template}

\begin{lstlisting}
\documentclass[template=ilcibernetico,bibfile=refs.bib]{zigma-class/zigma}
\end{lstlisting}

Features:
\begin{itemize}
  \item Green color scheme (\#009966)
  \item Rho base class
  \item IEEE bibliography
  \item Custom headers with lead author
  \item Journal-specific formatting
\end{itemize}

\section{Overriding Templates}

You can override template settings:

\begin{lstlisting}
\documentclass[template=ieee]{zigma-class/zigma}

\zigmasetup{
  % Override template defaults
  bib.style = {apa},          % Change from IEEE to APA
  color.main = {0000ff},      % Change color scheme
}
\end{lstlisting}

\begin{tcolorbox}[warningbox]
Template overrides generate warnings to alert you of conflicts. User settings always take precedence.
\end{tcolorbox}

\chapter{Customization}

\section{Colors}

\subsection{Main Color Scheme}

\begin{lstlisting}
\zigmasetup{
  color.main = {008080},     % Teal (default)
}
\end{lstlisting}

For rho base class:

\begin{lstlisting}
\zigmasetup{
  rhocolor = {009966},       % Green for rho
}
\end{lstlisting}

\section{Headers and Footers}

\subsection{Custom Headers}

\begin{lstlisting}
\zigmasetup{
  header.left = {My Custom Header},
  header.center = {\thepage},
  header.right = {Right Header},
}
\end{lstlisting}

\subsection{Custom Footers}

\begin{lstlisting}
\zigmasetup{
  footer.left = {Footer Left},
  footer.center = {Footer Center},
  footer.right = {Footer Right},
}
\end{lstlisting}

\begin{tcolorbox}[notebox]
Header/footer customization is fully supported in the rho base class. Other base classes use their native header/footer systems.
\end{tcolorbox}

\section{Layout Options}

Layout must be configured via class options (not \cmd{zigmasetup}):

\begin{lstlisting}
\documentclass[
  base=rho,
  marginleft=3cm,
  marginright=3cm,
  margintop=2.5cm,
  marginbottom=2.5cm,
  columnsep=25pt
]{zigma-class/zigma}
\end{lstlisting}

Available layout options:
\begin{itemize}
  \item \opt{marginleft} (default: 1.25cm for rho)
  \item \opt{marginright} (default: 1.25cm for rho)
  \item \opt{margintop} (default: 2cm for rho)
  \item \opt{marginbottom} (default: 2cm for rho)
  \item \opt{columnsep} (default: 15pt for rho)
\end{itemize}

\section{Page Numbering}

\begin{lstlisting}
\zigmasetup{
  page.start = {101},        % Start at page 101
}
\end{lstlisting}

Useful for journal articles in compiled volumes.

%% ====================================================================
%% PART IV: REFERENCE
%% ====================================================================

\part{Reference}

\chapter{Complete Key Reference}

\section{Document Metadata}

\begin{description}
  \item[\opt{title}] Document title (required)
  \item[\opt{subtitle}] Optional subtitle
  \item[\opt{date}] Publication date (default: \cmd{today})
  \item[\opt{date.show}] Boolean: show/hide date (default: true)
  \item[\opt{abstract}] Abstract text
  \item[\opt{keywords}] Comma-separated keywords
\end{description}

\section{Title URLs}

\begin{description}
  \item[\opt{title.url}] URL for clickable title
  \item[\opt{title.url.show}] Boolean: show URL (default: false)
\end{description}

\section{Journal Information}

\begin{description}
  \item[\opt{journal.name}] Journal name
  \item[\opt{journal.url}] Journal website URL
  \item[\opt{journal.url.show}] Boolean: show URL (default: false)
\end{description}

\section{Authors and Affiliations}

\begin{description}
  \item[\opt{affiliations.N}] Affiliation N (N = 0, 1, 2, ...)
  \item[\opt{authors.N.name}] Author N name (required)
  \item[\opt{authors.N.email}] Author N email
  \item[\opt{authors.N.orcid}] Author N ORCID identifier
  \item[\opt{authors.N.affiliation}] Comma-separated affiliation IDs
  \item[\opt{authors.N.corresponding}] Boolean: corresponding author
\end{description}

\section{Corresponding Author Settings}

\begin{description}
  \item[\opt{corresponding.marker}] Symbol: envelope, star, asterisk (default: envelope)
  \item[\opt{corresponding.show-footer}] Boolean: show footer (default: true)
\end{description}

\section{Metadata Footer}

\begin{description}
  \item[\opt{metadata.doi}] DOI identifier
  \item[\opt{metadata.url}] Custom URL
  \item[\opt{metadata.url.show}] Boolean: show URL (default: false)
  \item[\opt{metadata.received}] Received date (YYYY-MM-DD)
  \item[\opt{metadata.revised}] Revised date
  \item[\opt{metadata.accepted}] Accepted date
  \item[\opt{metadata.published}] Published date
  \item[\opt{metadata.license}] License text (e.g., "CC BY 4.0")
\end{description}

\section{Bibliography Settings}

\begin{description}
  \item[\opt{bib.style}] Citation style: ieee, apa, chicago, nature, harvard, vancouver, mla, alphabetic, verbose, etc. (15 total)
  \item[\opt{bib.backend}] Backend: biblatex or natbib (default: biblatex)
  \item[\opt{bib.sorting}] Boolean: enable sorting (default: true)
\end{description}

\section{Cross-Reference Labels}

All labels support singular and plural forms:

\begin{description}
  \item[\opt{labels.section / sections}] Section labels
  \item[\opt{labels.chapter / chapters}] Chapter labels
  \item[\opt{labels.figure / figures}] Figure labels
  \item[\opt{labels.table / tables}] Table labels
  \item[\opt{labels.equation / equations}] Equation labels
  \item[\opt{labels.listing / listings}] Listing labels
  \item[\opt{labels.algorithm / algorithms}] Algorithm labels
  \item[\opt{labels.theorem / theorems}] Theorem labels
  \item[\opt{labels.lemma / lemmas}] Lemma labels
  \item[\opt{labels.definition / definitions}] Definition labels
  \item[\opt{labels.appendix / appendices}] Appendix labels
  \item[\opt{labels.page / pages}] Page labels
\end{description}

\section{Header and Footer}

\begin{description}
  \item[\opt{header.left}] Left header content
  \item[\opt{header.center}] Center header content
  \item[\opt{header.right}] Right header content
  \item[\opt{footer.left}] Left footer content
  \item[\opt{footer.center}] Center footer content
  \item[\opt{footer.right}] Right footer content
\end{description}

\section{Colors}

\begin{description}
  \item[\opt{color.main}] Main color (hex, e.g., 008080)
  \item[\opt{rhocolor}] Rho base class color (hex)
\end{description}

\section{Page Settings}

\begin{description}
  \item[\opt{page.start}] Starting page number (integer)
\end{description}

\section{Debug Mode}

\begin{description}
  \item[\opt{debug}] Boolean: enable debug output (default: false)
\end{description}

\chapter{Command Reference}

\section{Setup Commands}

\begin{description}
  \item[\cmd{zigmasetup\{\meta{keys}\}}] Configure document settings
  \item[\cmd{maketitle}] Generate title, authors, abstract, keywords
\end{description}

\section{Cross-Reference Commands}

\begin{description}
  \item[\cmd{zigmaref\{\meta{label}\}}] Smart reference with auto-prefix
  \item[\cmd{Zigmaref\{\meta{label}\}}] Uppercase variant
  \item[\cmd{zigmaref*\{\meta{label}\}}] Number only (no prefix)
  \item[\cmd{zigmarefs\{\meta{labels}\}}] Multiple references
  \item[\cmd{zigmapageref\{\meta{label}\}}] Page reference
  \item[\cmd{zigmafullref\{\meta{label}\}}] Full reference (number + page)
\end{description}

\section{Citation Commands}

Standard biblatex/natbib commands work transparently:

\begin{description}
  \item[\cmd{cite\{\meta{key}\}}] Basic citation
  \item[\cmd{textcite\{\meta{key}\}}] In-text citation (biblatex)
  \item[\cmd{parencite\{\meta{key}\}}] Parenthetical citation (biblatex)
  \item[\cmd{footcite\{\meta{key}\}}] Footnote citation (biblatex)
  \item[\cmd{citep\{\meta{key}\}}] Parenthetical (natbib)
  \item[\cmd{citet\{\meta{key}\}}] Textual (natbib)
\end{description}

\section{Bibliography Printing}

\begin{description}
  \item[\cmd{printbibliography}] Print complete bibliography
  \item[\cmd{printbibliography[keyword=\meta{kw}]}] Filter by keyword
  \item[\cmd{printbibliography[title=\meta{title}]}] Custom heading
\end{description}

\chapter{Class Options Reference}

\section{Base Class Selection}

\begin{description}
  \item[\opt{base=\meta{name}}] Select base class: article, memoir, scrartcl, scrreprt, scrbook, rho (default: rho)
\end{description}

\section{Bibliography Options}

\begin{description}
  \item[\opt{bib}] Enable bibliography support
  \item[\opt{bibfile=\{\meta{file(s)}\}}] Specify .bib file(s) (comma-separated)
  \item[\opt{bibpreset=\meta{style}}] Citation style preset
  \item[\opt{bibbackend=\meta{backend}}] biblatex or natbib
\end{description}

\section{Template Options}

\begin{description}
  \item[\opt{template=\meta{name}}] Load template: ieee, apa, nature, thesis, ilcibernetico
\end{description}

\section{Layout Options}

\begin{description}
  \item[\opt{marginleft=\meta{dim}}] Left margin (e.g., 3cm)
  \item[\opt{marginright=\meta{dim}}] Right margin
  \item[\opt{margintop=\meta{dim}}] Top margin
  \item[\opt{marginbottom=\meta{dim}}] Bottom margin
  \item[\opt{columnsep=\meta{dim}}] Column separation (e.g., 25pt)
\end{description}

\begin{tcolorbox}[notebox]
Layout options must be specified as class options, not in \cmd{zigmasetup}, because geometry package requires early configuration.
\end{tcolorbox}

%% ====================================================================
%% PART V: APPENDICES
%% ====================================================================

\part{Appendices}

\appendix

\chapter{Complete Examples}

\section{Minimal Article}

\begin{lstlisting}
\documentclass{zigma-class/zigma}

\zigmasetup{
  title = {A Short Note on Quantum Computing},
  
  affiliations.0 = {MIT},
  
  authors.0.name = {Alice Johnson},
  authors.0.email = {alice@mit.edu},
  authors.0.affiliation = {0},
}

\begin{document}
\maketitle

\section{Introduction}

Your content here.

\end{document}
\end{lstlisting}

\section{Multi-Author Paper with Bibliography}

\begin{lstlisting}
\documentclass[bib,bibfile=refs.bib]{zigma-class/zigma}

\zigmasetup{
  title = {Machine Learning in Genomics},
  subtitle = {A Survey},
  abstract = {This paper surveys recent advances...},
  keywords = {machine learning, genomics, bioinformatics},
  
  affiliations.0 = {Stanford University},
  affiliations.1 = {Harvard Medical School},
  
  authors.0.name = {John Smith},
  authors.0.email = {john@stanford.edu},
  authors.0.orcid = {0000-0001-1111-1111},
  authors.0.affiliation = {0},
  authors.0.corresponding = {true},
  
  authors.1.name = {Jane Doe},
  authors.1.affiliation = {1},
  
  bib.style = {nature},
  
  metadata.doi = {10.1234/journal.2024.001},
  metadata.received = {2024-01-15},
  metadata.accepted = {2024-05-20},
}

\begin{document}
\maketitle

\section{Introduction}
\label{sec:intro}

Recent advances in machine learning \cite{smith2023} have...

\section{Methods}
\label{sec:methods}

As discussed in \zigmaref{sec:intro}, we apply...

\printbibliography

\end{document}
\end{lstlisting}

\section{PhD Thesis}

\begin{lstlisting}
\documentclass[base=memoir,template=thesis]{zigma-class/zigma}

\zigmasetup{
  title = {Advanced Topics in Quantum Field Theory},
  subtitle = {A Dissertation},
  
  affiliations.0 = {Department of Physics, MIT},
  
  authors.0.name = {Robert Brown},
  authors.0.affiliation = {0},
}

\begin{document}

\frontmatter
\maketitle
\tableofcontents

\mainmatter

\begin{refsection}
  \chapter{Introduction}
  \label{ch:intro}
  
  Content with citations \cite{ref1}...
  
  \printbibliography[heading=subbibliography]
\end{refsection}

\begin{refsection}
  \chapter{Quantum Electrodynamics}
  \label{ch:qed}
  
  Building on \zigmaref{ch:intro}, we now...
  
  \printbibliography[heading=subbibliography]
\end{refsection}

\end{document}
\end{lstlisting}

\chapter{Troubleshooting}

\section{Common Issues}

\subsection{Cross-References Show ??}

\textbf{Problem}: References display as "??" instead of numbers.

\textbf{Solution}: Run LuaLaTeX twice (or use latexmk):

\begin{lstlisting}[language=bash]
lualatex document.tex
lualatex document.tex
\end{lstlisting}

\subsection{Bibliography Not Appearing}

\textbf{Problem}: Bibliography section is empty.

\textbf{Solution}: Run the complete compilation sequence:

\begin{lstlisting}[language=bash]
lualatex document.tex
biber document        # or bibtex
lualatex document.tex
lualatex document.tex
\end{lstlisting}

Or use latexmk:

\begin{lstlisting}[language=bash]
latexmk -lualatex document.tex
\end{lstlisting}

\subsection{ORCID Icons Not Showing}

\textbf{Problem}: ORCID identifiers don't display icons.

\textbf{Solution}: Install the \pkg{academicons} package. On TeX Live:

\begin{lstlisting}[language=bash]
tlmgr install academicons
\end{lstlisting}

\subsection{Headers/Footers Not Working}

\textbf{Problem}: Custom headers/footers not applied.

\textbf{Solution}: Header/footer customization is fully supported only in rho base class. For other base classes, use their native mechanisms.

\subsection{Layout Options Ignored}

\textbf{Problem}: Margin settings in \cmd{zigmasetup} don't work.

\textbf{Solution}: Layout options must be class options, not \cmd{zigmasetup} keys:

\begin{lstlisting}
% Wrong:
\zigmasetup{marginleft=3cm}  % Doesn't work

% Correct:
\documentclass[marginleft=3cm]{zigma-class/zigma}
\end{lstlisting}

\section{Debug Mode}

Enable debug output to diagnose issues:

\begin{lstlisting}
\zigmasetup{
  debug = {true},
}
\end{lstlisting}

This prints diagnostic information to the console:
\begin{itemize}
  \item Key parsing details
  \item Author registration
  \item State dumps
  \item Module loading
\end{itemize}

\section{Getting Help}

\begin{itemize}
  \item \textbf{Repository}: \url{https://git.xed.it/exedre/zigma-class}
  \item \textbf{Issues}: \url{https://git.xed.it/exedre/zigma-class/issues}
  \item \textbf{Email}: \texttt{emmanuele@exedre.org}
\end{itemize}

\chapter{Version History}

\section{Version 0.9.0 (November 7, 2025)}

\textbf{Major Release}: Smart Cross-Reference System

\begin{itemize}
  \item Smart cross-reference system with auto-detection
  \item 11 supported label types
  \item Multiple reference support with Oxford comma
  \item Multilingual label support (23 new keys)
  \item 15 total citation styles
  \item Complete documentation and test coverage
\end{itemize}

\section{Version 0.8.7 (November 7, 2025)}

\textbf{Citation Styles Expansion}

\begin{itemize}
  \item Added 8 new citation styles
  \item Total: 15 citation styles with complete documentation
  \item Philosophy citation style
\end{itemize}

\section{Version 0.8.6 (November 7, 2025)}

\textbf{Major Refactoring}

\begin{itemize}
  \item Renamed plugins → baseclasses
  \item Templates reorganization
  \item Test suite rationalization (47 tests)
  \item Integration guides (Zotero, Mendeley)
\end{itemize}

\section{Version 0.8.5 (November 7, 2025)}

\textbf{Bibliography Integration}

\begin{itemize}
  \item Complete bibliography support
  \item 6 initial citation presets
  \item Per-chapter bibliographies
  \item Split bibliographies by keyword
  \item Multiple .bib file support
\end{itemize}

\section{Earlier Versions}

\begin{description}
  \item[v0.8.0] Metadata footer system, clickable titles
  \item[v0.7.0] Complete plugin system with 4 plugins
  \item[v0.6.x] ORCID integration, corresponding markers, memoir plugin, KOMA plugin
  \item[v0.5.0] Core system with author management
\end{description}

For complete version history, see \texttt{CHANGELOG.md} in the repository.

\chapter{Future Roadmap}

\section{Version 1.0.0 (Target: Q2 2026)}

CTAN Release:

\begin{itemize}
  \item TDS (TeX Directory Structure) packaging
  \item Complete PDF manual (this document)
  \item Polished example collection
  \item Installation documentation
  \item CTAN submission
\end{itemize}

\section{Beyond v1.0.0}

Potential future features:

\begin{itemize}
  \item Additional base classes (beamer, revtex)
  \item Enhanced validation system
  \item Performance optimizations
  \item Additional templates
  \item Community contributions
\end{itemize}

%% ====================================================================
%% INDEX
%% ====================================================================

\backmatter

\chapter*{Index}
\addcontentsline{toc}{chapter}{Index}

\begin{multicols}{2}
\small

\textbf{A}

Abstract, \pageref{sec:intro}\\
Academicons package, 23\\
Affiliations, 34--35\\
APA style, 51\\
Authors, 34--38

\bigskip

\textbf{B}

Base classes, 76--80\\
Bibliography, 47--59\\
— multiple files, 53\\
— per-chapter, 54\\
— split, 55\\
Biblatex, 48

\bigskip

\textbf{C}

Chicago style, 51\\
Citations, 48--52\\
Class options, 110--111\\
Colors, 89\\
Commands, 107--109\\
Configuration, 30--33\\
Corresponding authors, 37--38\\
Cross-references, 60--75

\bigskip

\textbf{D}

Debug mode, 117\\
DOI, 41--42

\bigskip

\textbf{E}

Examples, 112--114

\bigskip

\textbf{H}

Harvard style, 51\\
Headers and footers, 88--89

\bigskip

\textbf{I}

IEEE style, 51\\
Installation, 16--17

\bigskip

\textbf{K}

Keywords, 32\\
KOMA-Script, 79

\bigskip

\textbf{L}

Labels, 60--75\\
Layout options, 90, 111\\
License, 2

\bigskip

\textbf{M}

Memoir base class, 78\\
Metadata, 41--46\\
MLA style, 51\\
Multilingual support, 71--73

\bigskip

\textbf{N}

Nature style, 51

\bigskip

\textbf{O}

ORCID, 36

\bigskip

\textbf{P}

Page numbering, 90\\
Philosophy style, 51

\bigskip

\textbf{R}

References, 60--75\\
Requirements, 15\\
Rho base class, 80

\bigskip

\textbf{S}

Setup command, 30

\bigskip

\textbf{T}

Templates, 81--87\\
Title page, 30--33\\
Troubleshooting, 115--117

\bigskip

\textbf{V}

Vancouver style, 51\\
Version history, 118--119

\bigskip

\textbf{Z}

Zigmaref, 60--70\\
Zigmasetup, 30

\end{multicols}

\end{document}
